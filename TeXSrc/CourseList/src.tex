\documentclass[11pt]{article}
%\documentclass{article}
\usepackage[letterpaper,margin=1in]{geometry}
\usepackage{xcolor}
\usepackage{fancyhdr}

% Fonts, so hard to choose..
%\usepackage{tgschola} 
%\usepackage{baskervald} 
%\usepackage{kpfonts}
%\usepackage[libertine,cmintegrals,cmbraces,vvarbb]{newtxmath}

\pagestyle{fancy}
\fancyhf{}
\fancyhead[C]{%
  \footnotesize\sffamily
  \yourname\quad
  Website: \textcolor{blue}{\itshape\yourweb}\quad
  \textcolor{blue}{\youremail}}

\newcommand{\soptitle}{Course List}
\newcommand{\yourname}{Abhijit Chowdhary}
\newcommand{\youremail}{ac6361@nyu.edu}
\newcommand{\yourweb}{\url{https://abhijit-c.github.io/}}

\newcommand{\statement}[1]{\par\medskip
  \underline{\textcolor{blue}{\textbf{#1:}}}\space
}

\usepackage[
  colorlinks,
  breaklinks,
  pdftitle={\yourname - \soptitle},
  pdfauthor={\yourname},
  unicode
]{hyperref}

\begin{document}

\begin{center}\LARGE\soptitle\\
\large of \yourname\ (Applied Mathematics Ph.D. applicant for Fall---2020)
\end{center}

\hrule
\vspace{1pt}
\hrule height 1pt

\bigskip

\section*{Summer 2016}

{\bf Data Structures:} Use and design of data structures, which organize
information in computer memory. Stacks, queues, linked lists, binary trees: how
to implement them in a high-level language, how to analyze their effect on
algorithm efficiency, and how to modify them. Programming assignments.

\section*{Fall 2016}

{\bf Computer Systems Organization:} Covers the internal structure of computers,
machine (assembly) language programming, and the use of pointers in high-level
languages. Topics include the logical design of computers, computer
architecture, the internal representation of data, instruction sets, and
addressing logic, as well as pointers, structures, and other features of
high-level languages that relate to assembly language. Programming assignments
are in both assembly language and other languages.

{\bf Honors Linear Algebra:} This honors section of Linear Algebra is
a proof-based course intended for well-prepared students who have already
developed some mathematical maturity and ease with abstraction. Its scope will
include the usual Linear Algebra (MATH-UA 140) syllabus; however this class will
be faster, more abstract and proof-based, covering additional topics. Topics
covered are: Vector spaces, linear dependence, basis and dimension, matrices,
determinants, solving linear equations, linear transformations, eigenvalues and
eigenvectors, diagonalization, inner products, applications.

\section*{Spring 2017}

{\bf Operating Systems:} Covers the principles and design of operating systems.
Topics include process scheduling and synchronization, deadlocks, memory
management (including virtual memory), input/output, and file systems.
Programming assignments.

{\bf Basic Algorithms:} Introduction to the study of algorithms. Presents two
main themes: designing appropriate data structures and analyzing the efficiency
of the algorithms that use them. Algorithms studied include sorting, searching,
graph algorithms, and maintaining dynamic data structures. Homework assignments,
not necessarily involving programming.

{\bf Numerical Computing:} The need for floating-point arithmetic, the IEEE
floating-point standard, and the importance of numerical computing in a wide
variety of scientific applications. Fundamental types of numerical algorithms:
direct methods (e.g., for systems of linear equations), iterative methods (e.g.,
for a nonlinear equation), and discretization methods (e.g., for a differential
equation). Numerical errors: can you trust your answers? Uses graphics and
software packages such as Matlab. Programming assignments.

\section*{Summer 2017}

{\bf Complex Variables for Scientists and Engineers:} This course is an
introduction to complex variables accessible to juniors and seniors in
engineering, physics and mathematics. The algebra of complex numbers, analytic
functions, Cauchy Integral Formula, theory of residues and application to the
evaluation of real integrals, conformal mapping and applications to physical
problems.

{\bf Introduction to Number Theory:} Integers, divisibility, prime numbers,
unique factorization, congruences, quadratic reciprocity, Diophantine equations
and arithmetic functions.

\section*{Fall 2017}

{\bf Honors Analysis I:} This is an introduction to the rigorous treatment of
the foundations of real analysis in one variable. It is based entirely on
proofs. Students are expected to know what a mathematical proof is and are also
expected to be able to read a proof before taking this class. Topics include:
properties of the real number system, sequences, continuous functions, topology
of the real line, compactness, derivatives, the Riemann integral, sequences of
functions, uniform convergence, infinite series and Fourier series. Additional
topics may include: Lebesgue measure and integral on the real line, metric
spaces, and analysis on metric spaces.

{\bf Honors Algebra I:} Introduction to abstract algebraic structures, including
groups, rings, and fields. Sets and relations. Congruences and unique
factorization of integers. Groups, permutation groups, group actions,
homomorphisms and quotient groups, direct products, classification of finitely
generated abelian groups, Sylow theorems. Rings, ideals and quotient rings,
Euclidean rings, polynomial rings, unique factorization.

{\bf Numerical Methods I:} This course is part of a two-course series meant to
introduce graduate students in mathematics to the fundamentals of numerical
mathematics (but any Ph.D. student seriously interested in applied mathematics
should take it). It will be a demanding course covering a broad range of topics.
There will be extensive homework assignments involving a mix of theory and
computational experiments, and an in-class final. Topics covered in the class
include floating-point arithmetic, solving large linear systems, eigenvalue
problems, interpolation and quadrature (approximation theory), nonlinear systems
of equations, linear and nonlinear least squares, nonlinear optimization, and
Fourier transforms. This course will not cover differential equations, which
form the core of the second part of this series, Numerical Methods II.

\section*{Spring 2018}

{\bf Honors Analysis II:} This is a continuation of MATH-UA 328 Honors Analysis
I. Topics include: metric spaces, differentiation of functions of several real
variables, the implicit and inverse function theorems, Riemann integral on Rn,
Lebesgue measure on Rn, the Lebesgue integral.

{\bf Honors Algebra II:} Principle ideal domains, polynomial rings in several
variables, unique factorization domains. Fields, finite extensions,
constructions with ruler and compass, Galois theory, solvability by radicals.

{\bf Topology:} Set-theoretic preliminaries. Metric spaces, topological spaces,
compactness, connectedness, covering spaces, and homotopy groups.

{\bf Special Topics: Geometric Modeling:} Recent advances in 3D digital geometry
processing have created a plenitude of novel concepts for the mathematical
representation and interactive manipulation of geometric models. This course
covers some of the latest developments in geometric modeling and digital
geometry processing. Topics include surface modeling based on polygonal meshes,
surface reconstruction, mesh improvement, mesh parametrization, discrete
differential geometry, interactive shape editing, skinning animation,
architectural and structure-aware geometric modeling, shape modeling with an eye
on 3D printing.  The students will learn how to design, program and analyze
algorithms and systems for interactive 3D shape modeling and digital geometry
processing.

\section*{Summer 2018}

{\bf Introduction to Artificial Intelligence:} Introduces a range of ideas and
methods in AI, varying semester to semester but chosen largely from: automated
heuristic search, planning, games, knowledge representation, logical and
statistical inference, learning, natural language processing, vision, robotics,
cognitive modeling, and intelligent agents. Programming projects will help
students obtain a hands-on feel for various topics.

{\bf Partial Differential Equations:} Introduction to the subject of partial
differential equations: first order equations (linear and nonlinear), heat
equation, wave equation, and Laplace equation. Examples of nonlinear equations
of each type. Qualitative properties of solutions. Method of characteristics for
hyperbolic problems. Solution of initial boundary value problems using
separation of variables and eigenfunction expansions. Some numerical methods.

\section*{Fall 2018}

{\bf Partial Differential Equations:} A basic introduction to PDEs, designed for
a broad range of students whose goals may range from theory to applications.
This course emphasizes examples, representation formulas, and properties that
can be understood using relatively elementary tools. We will take a broad
viewpoint, including how the equations we consider emerge from applications, and
how they can be solved numerically. Topics will include: the heat equation; the
wave equation; Laplace's equation; conservation laws; and Hamilton-Jacobi
equations. Methods introduced through these topics will include: fundamental
solutions and Green's functions; energy principles; maximum principles;
separation of variables; Duhamel's principle; the method of characteristics;
numerical schemes involving finite differences or Galerkin approximation; and
many more.

{\bf Algebra I:} Basic concepts of groups, rings and fields. Symmetry groups,
linear groups, Sylow theorems; quotient rings, polynomial rings, ideals, unique
factorization, Nullstellensatz; field extensions, finite fields.

{\bf Honors Theory of Probability:} Counting, Permutations and Combinations,
Uncertainty and Probability. Calculating probabilities. Independence
, conditional probability. Some discrete distributions. random variables,
expectation and variance. Generating functions.  Joint distributions.
Covariance and correlation. Law of large numbers. Continuous distributions.
Normal distribution. Central limit theorem.  Charateristic functions. Inversion
Theorem. Continuity Theorem.  Multivariate distributions. Change of variables.
Multivariate Normal and related distributions. Poisson Processes. Markov Chains.
Brownian Motion.

\section*{Spring 2019}

{\bf Numerical Methods II:} This course will cover fundamental methods that are
essential for the numerical solution of differential equations. It is intended
for students familiar with ODE and PDE and interested in numerical computing;
computer programming assignments in MATLAB will form an essential part of the
course. The course will introduce students to numerical methods for (1) ordinary
differential equations, explicit and implicit Runge-Kutta and multistep methods,
convergence and stability; (2) finite difference and finite element and integral
equation methods for elliptic partial differential equations (Poisson eq.); (4)
spectral methods and the FFT, exponential temporal integrators, and multigrid
iterative solvers; and (5) finite difference and finite volume parabolic
(diffusion/heat eq.) and hyperbolic (advection and wave) partial differential
equations.

{\bf Chaos and Dynamical Systems:} Topics will include dynamics of maps and of
first order and second-order differential equations, stability, bifurcations,
limit cycles, dissection of systems with fast and slow time scales.  Geometric
viewpoint, including phase planes, will be stressed. Chaotic behavior will be
introduced in the context of one-variable maps (the logistic), fractal sets,
etc. Applications will be drawn from physics and biology. There will be homework
and projects, and a few computer lab sessions (programming experience is not
a prerequisite).

{\bf Advanced Topics In Numerical Analysis: High Performance Computing:} This
class will be an introduction to the fundamentals of parallel scientific
computing. We will establish a basic understanding of modern computer
architectures (CPUs and accelerators, memory hierarchies, interconnects) and of
parallel approaches to programming these machines (distributed vs. shared memory
parallelism: MPI, OpenMP, OpenCL/CUDA). Issues such as load balancing,
communication, and synchronization will be covered and illustrated in the
context of parallel numerical algorithms. Since a prerequisite for good parallel
performance is good serial performance, this aspect will also be addressed.
Along the way you will be exposed to important tools for high performance
computing such as debuggers, schedulers, visualization, and version control
systems. This will be a hands-on class, with several parallel (and serial)
computing assignments, in which you will explore material by yourself and try
things out. There will be a larger final project at the end. You will learn some
Unix in this course, if you don't know it already. Prerequisites for the course
are (serial) programming experience with C/C++ (I will use C in class) or
FORTRAN, and some familiarity with numerical methods.

\section*{Fall 2019}

{\bf Advanced Topics In Numerical Methods: Finite Element Methods:} This course
covers theoretical and practical aspects of finite element methods for the
numerical solution of partial differential equations. The first part of the
course will focus on theoretical foundations of the method (calculus of
variations, Poincare inequality, Cea's lemma, Nitsche trick, convergence
estimates). The second part targets practical aspects of the method, illustrates
how it can be implemented and used for solving partial differential equations in
two and three dimensions. Examples will include the Poisson equation, linear
elasticity and, time permitting, the Stokes equations.

{\bf Methods of Applied Math:} This is a first-year course for all incoming PhD
and Masters students interested in pursuing research in applied mathematics. It
provides a concise and self-contained introduction to advanced mathematical
methods, especially in the asymptotic analysis of differential equations. Topics
include scaling, perturbation methods, multi-scale asymptotics, transform
methods, geometric wave theory, and calculus of variations

\section*{Intended in Spring 2020}

{\bf Advanced Topics In Numerical Analysis: Nonsmooth Optimization:} Convex
optimization problems have many important properties, including a powerful
duality theory and the property that any local minimum is also a global minimum.
Nonsmooth optimization refers to minimization of functions that are not
necessarily convex, usually locally Lipschitz, and typically not differentiable
at their minimizers. Topics in convex optimization that will be covered include
duality, linear and semidefinite programming, CVX ("disciplined convex
programming"), gradient and Newton methods, Nesterov's complexity bound, the
alternating direction method of multipliers, the nuclear norm and matrix
completion, the primal barrier method, primal-dual interior-point methods for
linear and semidefinite programs. Topics in nonsmooth optimization that will be
covered include subgradients and subdifferentials, Clarke regularity, and
algorithms, including gradient sampling and BFGS, for nonsmooth, nonconvex
optimization. Homework will be assigned, both mathematical and computational.
Students may submit a final project on a pre-approved topic or take a written
final exam.

{\bf Basic Probability:} The one-semester course introduces the basic concepts
and methods of probability. Topics include: probability spaces, random
variables, distributions, law of large numbers, central limit theorem, random
walk martingales in discrete time, and if time permits Markov chains and
Brownian motion.

{\bf Mechanics:} This course provides brief mathematical introductions to
elasticity, classical mechanics, and statistical mechanics -- topics at the
interface where differential equations and probability meet physics and
materials science. For students preparing to do research on physical
applications, the class provides an introduction to crucial concepts and tools;
for students of analysis the class provides valuable context by exploring some
central applications. No prior exposure to mechanics or physics is assumed.  The
segment on elasticity (about 6 weeks) will include: one-dimensional models
(strings and rods); buckling as a bifurcation; nonlinear elasticity for 3D
solids; and linear elasticity. The segment on classical mechanics (about
5 weeks) will include: basic examples; alternative formulations including action
minimization and Hamilton's equations; relations to the Calculus of Variations
including Hamilton-Jacobi equations, optimal control, and geodesics; stability
and parametric resonance. The segment on statistical mechanics (about 3 weeks)
will include basic concepts such as the microcanonical and canonical ensembles,
entropy, and the equilibrium distribution; some simple examples; and the
numerical method known as Metropolis sampling.



\end{document}
