%\documentclass[12pt]{article}
\documentclass{article}
\usepackage[letterpaper,margin=1in]{geometry}
\usepackage{xcolor}
\usepackage{fancyhdr}

% Fonts, so hard to choose..
%\usepackage{tgschola} 
%\usepackage{baskervald} 
%\usepackage{kpfonts}
%\usepackage[libertine,cmintegrals,cmbraces,vvarbb]{newtxmath}

\pagestyle{fancy}
\fancyhf{}
\fancyhead[C]{%
  \footnotesize\sffamily
  \yourname\quad
  Website: \textcolor{blue}{\itshape\yourweb}\quad
  \textcolor{blue}{\youremail}}

\newcommand{\soptitle}{Statement of Purpose}
\newcommand{\yourname}{Abhijit Chowdhary}
\newcommand{\youremail}{ac6361@nyu.edu}
\newcommand{\yourweb}{\url{https://abhijit-c.github.io/}}

\newcommand{\statement}[1]{\par\medskip
  \underline{\textcolor{blue}{\textbf{#1:}}}\space
}

\usepackage[
  colorlinks,
  breaklinks,
  pdftitle={\yourname - \soptitle},
  pdfauthor={\yourname},
  unicode
]{hyperref}

\begin{document}

\begin{center}\LARGE\soptitle\\
\large of \yourname\ (Applied Mathematics Ph.D. applicant for Fall---2020)
\end{center}

\hrule
\vspace{1pt}
\hrule height 1pt

\bigskip

\textbf{Note to my letter writers:} This document is a generic statement of purpose
template. It contains the main body of what I'll be presenting in my tailored
statement of purposes, which will be adapted slightly depending on the program
and university I'm applying to. So for example, in places where I'm applying for
computational math, this is good as it stands, but for the CS applications,
naturally I'll take a more computational bent, pure dynamical systems the
opposite, etc. Also note that Github's pdf viewer doesn't allow one to click on
the linked paper's / projects in my statement (highlighted in pink), but if you
download the pdf, you can follow the links to the referenced documents.

\vspace{5pt}

I must admit, coming into college, let alone liking computational mathematics,
I hadn't even heard of its existence. I began as a computer science major,
having held hobbies in computers throughout high school, and like all typical
incoming freshman, was advised to take linear algebra. At this point, having
just learned how to code well, I itched to apply my skills and after being
handed a particularly wail-inducing determinant calculation in one of my
problem sets, I instead decided to write a script. When I went to office hours to
gloat, he pointed to the \textit{Numerical Recipes} text by Press on his shelf;
I think that's the moment when I was doomed to love computational math. Although
my motivations have matured from those of a lazy freshman, my goal since then
has remained the same; I want to become an academic in order to continue to
thoroughly enjoy and immerse myself in my research interests. To do so, my
second step after my undergraduate education is a Ph.D., which is why I eagerly
apply to this program.

After that semester, I had madly dedicated myself to improving my then lacking
mathematical ability. Computational mathematics is a field which demands a huge
breadth of background in various different subjects, so I intentionally loaded
my schedule, took summer courses, and from my sophomore year onward I made sure
to take advantage of the accelerated and more detailed graduate courses here at
NYU. At the end of my undergraduate career here at NYU, I'll have taken 11
graduate courses in computer science and math, including finishing the
introductory numerical analysis sequence offered for incoming Ph.D. students and
three advanced topic seminars in computational math topics. In addition, in
order to keep my algorithmic skills sharp, I maintained a department tutor
position for both the undergraduate and graduate algorithms courses through four
semesters, stopping this semester only because my usual professor went on
sabbatical.

Although I took many courses, regretfully it didn't leave me much time to branch
out for research. However, I do have a few projects and original research under
my belt. In my geometric modeling course sophomore year, I investigated and
implemented the paper
\href{https://abhijit-c.github.io/storage/Guennebaud07.pdf}
{Algebraic Point Set Surfaces} 
by Gunnebaud and Gross, exploring and attempting to optimize the matrix assembly
and generalized eigenvalue problem underlying the optimization problem. In my
junior year, I wrote a topic paper on
\href{https://abhijit-c.github.io/Research/resources/Parareal/Parareal.pdf}
{Parareal},
a parallel finite difference scheme, discussing the details of its
implementation, strong and weak scaling properties, and its convergence,
efficiency, and stability analysis. Finally, the most formal research experience
I've had is in collaborating with Qiliang Wu from Ohio University and another
student in the summer of 2019 in an REU on the diffusive stability of the
Swift-Hohenberg equation near the Zigzag boundary in 2D. Originally, I was lured
into the REU with the promise of a numerical project, but was swiftly distracted
onto this topic instead, and despite it being out of my field at the time, the
beautiful analysis within the proof we managed to construct nearing the end of
project has convinced me to take a closer look at the theoretical approaches to
understanding the behavior of troublesome systems; it's why I'm taking a course
in pertubative methods now.  My favorite part of the program was it's freer
structure, as our days would be split into two halves, in the first of which we
would research, \LaTeX, and push the proof on our own, and in the second part we
would gather and try and work out the major difficulties of the proof together.
Through this, the REU convinced me that being a researcher is something that I'd
not only desire but be good at; the process of going to sleep and waking up with
a paper in hand and spending hours locked in a room with a problem and my
colleagues was invigorating to the limit. Finally, although the paper is not yet
in preprint, it's in the final draft stages and
\href{https://abhijit-c.github.io/Research/resources/SHE/slides.pdf}
{Professor Wu presented the proved result} 
at Miami Universities' recent conference in Differential Equations and Dynamical
Systems.

Thus, through these experiences of mine, I've narrowed down my research
interests to a couple of sub-fields. Specifically, I would like to study 
the numerical approximation of the solutions to PDEs and the theoretical
analysis on such methods, as well as the more theoretical dynamical systems and
pertubative approach to the analysis on nonlinear PDEs. I've also found lots of
interest in the area of high performance and parallel scientific computing,
having been sucked in during the studies of the former research interests.
Finally, I also have interests in computational number theory and algorithms in
algebraic geometry, which explains the otherwise seemingly random foray into
algebra on my transcript. During graduate school, I hope to be able make the
transition, as I did in my REU, from learning about these subjects and writing
survey papers on their topics to producing my own original ideas on the field.

\end{document}
