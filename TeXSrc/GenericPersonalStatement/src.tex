%\documentclass[12pt]{article}
\documentclass{article}
\usepackage[letterpaper,margin=1in]{geometry}
\usepackage{xcolor}
\usepackage{fancyhdr}

% Fonts, so hard to choose..
%\usepackage{tgschola} 
%\usepackage{baskervald} 
%\usepackage{kpfonts}
%\usepackage[libertine,cmintegrals,cmbraces,vvarbb]{newtxmath}

\pagestyle{fancy}
\fancyhf{}
\fancyhead[C]{%
  \footnotesize\sffamily
  \yourname\quad
  Website: \textcolor{blue}{\itshape\yourweb}\quad
  \textcolor{blue}{\youremail}}

\newcommand{\soptitle}{Statement of Purpose}
\newcommand{\yourname}{Abhijit Chowdhary}
\newcommand{\youremail}{ac6361@nyu.edu}
\newcommand{\yourweb}{\url{https://abhijit-c.github.io/}}

\newcommand{\statement}[1]{\par\medskip
  \underline{\textcolor{blue}{\textbf{#1:}}}\space
}

\usepackage[
  colorlinks,
  breaklinks,
  pdftitle={\yourname - \soptitle},
  pdfauthor={\yourname},
  unicode
]{hyperref}

\begin{document}

\begin{center}\LARGE\soptitle\\
\large of \yourname\ (Applied Mathematics PhD applicant for Fall---2020)
\end{center}

\hrule
\vspace{1pt}
\hrule height 1pt

\bigskip

I must admit, coming into college, let alone liking computational mathematics,
I didn't even know the field existed. I began as a computer science major,
having held hobbies in computers throughout high school, and like all typical
incoming freshman, was advised to take linear algebra. At this point, having
just learned how to code well, I itched to apply my skills and after being
handed a particularily wail-inducing determinant calculation in one of my
homeworks, I instead decided to write a script. When I went to office hours to
gloat, he pointed to the \textit{Numerical Recipies} text by Press on his shelf;
I think that's the moment when I was doomed to love computational math.

After then, I madly dedicated myself to improving my then lacking mathematical
ability. Computational mathematics is a field which demands a huge breadth of
background in various different subjects, so I intentionally loaded my schedule,
took summer courses, and from my sophomore year onward I made sure to take
advantage of the accelerated and more detailed graduate courses here at NYU. At
the end of my undergraduate career here at NYU, I'll have taken 11 graduate
courses in computer science and math, including finishing the introductory
numerical analysis sequence offered for incoming PhD students and three advanced
topic seminars in computational math topics. In addition, in order to keep my
algorithmic skills sharp, I maintained a department tutor position for both the
undergraduate and graduate algorithms courses through four semesters, stopping
this semester only because my usual professor went on sabatical.

Although I took many courses, regretfully it didn't leave me much time to branch
out for research. However, I do have a few projects and original research under
my belt. In my geometric modeling course sophomore year, I investigated and
implemented the paper
\href{https://abhijit-c.github.io/storage/Guennebaud07.pdf}
{Algebraic Point Set Surfaces} 
by Gunnebaud and Gross, exploring and attempting to optimize the matrix assembly
and generalized eigenvalue problem underlying the optimization problem. In my
junior year, I wrote a topic paper on
\href{https://abhijit-c.github.io/Research/resources/Parareal/Parareal.pdf}
{Parareal},
a parallel finite difference scheme, discussing the details of its
implementation, strong and weak scaling properties, and its convergence,
efficiency, and stability analysis. Finally, the most formal research experience
I've had is in collaborating with Qiliang Wu from Ohio University in the summer
of 2019 in an REU on the diffusive stability of the Swift-Hohenberg equation
near the Zigzag boundary in 2D. I was lured into the REU with the promise of
a numerical project, but was swiftly distracted onto this instead, and despite
it being out of my field at the time, the beautiful analysis within the proof we
managed to construct nearing the end of project has convinced me to take
a closer look at the theoretical approaches to understanding the behavior of
troublesome systems. Although the paper is in it's final draft stages, 
\href{https://abhijit-c.github.io/Research/resources/SHE/slides.pdf}
{Professor Wu presented the result} 
at Miami Universities' recent conference in Differential Equations and Dynamical
Systems.

Finally, through these experiences of mine, I've narrowed down my research
interests to a couple of subfields. Specifically, I would like to study 
the numerical approximation of the solutions to PDEs and the theoretical
analysis on such methods, as well as the more theoretical dynamical systems and
pertubative approach to the analysis on nonlinear PDEs. I've also found lots of
interest in the area of high performance and parallel scientific computing,
having been sucked in during the studies of the former research interests. My

\end{document}
