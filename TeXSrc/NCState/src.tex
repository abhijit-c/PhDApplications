\documentclass[11pt]{article}
%\documentclass{article}
\usepackage[letterpaper,margin=1in]{geometry}
\usepackage{xcolor}
\usepackage{fancyhdr}

% Fonts, so hard to choose..
%\usepackage{tgschola} 
%\usepackage{baskervald} 
%\usepackage{kpfonts}
%\usepackage[libertine,cmintegrals,cmbraces,vvarbb]{newtxmath}

\pagestyle{fancy}
\fancyhf{}
\fancyhead[C]{%
  \footnotesize\sffamily
  \yourname\quad
  Website: \textcolor{blue}{\itshape\yourweb}\quad
  \textcolor{blue}{\youremail}}

\newcommand{\soptitle}{Statement of Purpose}
\newcommand{\yourname}{Abhijit Chowdhary}
\newcommand{\youremail}{ac6361@nyu.edu}
\newcommand{\yourweb}{\url{https://abhijit-c.github.io/}}

\newcommand{\statement}[1]{\par\medskip
  \underline{\textcolor{blue}{\textbf{#1:}}}\space
}

\usepackage[
  colorlinks,
  breaklinks,
  pdftitle={\yourname - \soptitle},
  pdfauthor={\yourname},
  unicode
]{hyperref}

\begin{document}

\begin{center}\LARGE\soptitle\\
\large of \yourname\ (Applied Mathematics Ph.D. applicant for Fall---2020)
\end{center}

\hrule
\vspace{1pt}
\hrule height 1pt

\bigskip

I am intent on obtaining a Ph.D. in order to study and, later under
a university, research the field of computational mathematics. Currently my
main research interest is in the theory and practice of the {\em numerical
solution to PDEs}, especially under the high performance and parallel context.
In addition, I have more theoretical interests in the behavior of PDEs in a {\em
nonlinear dynamical systems} setting, specifically in approximating their
solutions in order to reveal underlying control parameters. Finally, explaining
my otherwise seemingly random foray into graduate algebra on my transcript,
I have interests in the algorithms of {\em symbolic computing}, especially as it
pertains to exact numerical computation and finite field algorithms. In short,
I'd like to apply my favored tool, computers, to my passion in math.

I began to have an inkling towards these interests early in my undergraduate
career, and since then I have madly dedicated myself to learning the field;
which by it's nature demands a huge breadth of background in various different
subjects. I intentionally loaded my schedule, took summer courses, and from my
sophomore year onward I made sure to take advantage of the accelerated and more
detailed graduate courses here at NYU. At the end of my undergraduate career
here at NYU, I'll have taken 11 graduate courses (out of 25 including
undergraduate \& summer courses) in computer science and math, including
finishing the numerical analysis sequence offered for incoming Ph.D. students,
three advanced topic seminars in computational math topics, and graduate PDE and
algebra to improve my foundations. In addition, in order to keep my algorithmic
skills sharp, I maintained a department tutor position for both the
undergraduate and graduate algorithms courses through four semesters, stopping
this semester only because my usual professor went on sabbatical. Also, in an
effort to try and evolve my maturity towards research, I occasionally ran
a computer science reading group through our local ACM chapter, which I am
a board member of, looking over more classical papers like the original
tit-for-tat iterated prisoners dilemma article, the cyclotomic AKS primality
test paper, etc. In addition, I have a few completed projects and original
research under my belt.

{\bf Algebraic Point Set Surfaces:} I took the graduate course geometric
modeling in my sophomore year, which investigated the cutting edge research in
the field. For the final project, I investigated and implemented the paper {\em
Algebraic Point Set Surfaces}%
\footnote{\url{https://abhijit-c.github.io/storage/Guennebaud07.pdf}}
by Gunnebaud and Gross, which was a modification
to the usual point set surface technique for construction of surface normals and
a mesh from a point cloud. I attempted to improve the matrix assembly via
parallelization and understand methods to improve the efficiency of the
generalized eigenvalue solve underlying the surface normal reconstruction.

{\bf Parareal:} In my junior year, I wrote a topic paper on {\em Parareal}%
\footnote{\url{https://abhijit-c.github.io/Research/resources/Parareal/Parareal.pdf}},
a parallel-in-time finite difference scheme, discussing the details of its
implementation, strong and weak scaling properties, and its convergence,
efficiency, and stability analysis. My implementation was tested on the high
performance computing cluster here at NYU, named Prince.

{\bf REU @ Ohio University:} Finally, the most formal research experience I've
had is in collaborating with Qiliang Wu from Ohio University and another student
in the summer of 2019 in an REU on the diffusive stability of the
Swift-Hohenberg equation near the Zigzag boundary in 2D. Originally, I was lured
into the REU with the promise of a numerical project, but was swiftly distracted
onto this topic instead, and despite it being out of my field at the time, the
beautiful analysis within the proof we managed to construct nearing the end of
project has convinced me to take a closer look at the theoretical approaches to
understanding the behavior of troublesome systems; it's why I'm taking a course
in perturbation theory now.  My favorite part of the program was it's freer
structure, as our days would be split into two halves, in the first of which we
would research, \LaTeX, and push the proof on our own, and in the second part we
would gather and try and work out the major difficulties of the proof together.
Through this, the REU convinced me that being a researcher is something that I'd
not only desire but be good at; the process of going to sleep and waking up with
a paper in hand and spending hours locked in a room with a problem and my
colleagues was invigorating to the limit. My main contributions to this project,
aside from those implicitly made in the group work, were in finding and
subsequently correcting an error in an old classical paper in the literature,
resulting in an appendix, the painstaking computation and verification (via
Mathematica) of a critical eigenvalue expansion, and in working out the final
details of the contraction argument, completing the proof.
Although the paper is not yet in preprint, it's in the final draft stages and
Professor Wu presented the research%
\footnote{\url{https://abhijit-c.github.io/Research/resources/SHE/slides.pdf}}
at Miami University's recent conference in Differential Equations and Dynamical
Systems.

During graduate school, I hope to be able make the transition, as I did in my
REU, from learning about these subjects and writing survey papers on their
topics, to producing my own original ideas on the field. To this end, I'm very
excited to apply to North Carolina State, one of the stronger schools in all
three of my interests. In addition to having a computational mathematics
concentration, which I will note is almost tailor made for someone like me, lots
of the department also researches on topics similar to my interest. To name
a few; towards numerical methods, I find {\em Professor Chertock's} recent work
in operator splitting based methods and difference schemes built for small
scales (asymptotic preserving methods) very exciting. In addition, having sat in
a computational inverse problems course, I find that {\em Professor Saibaba and
Professor Alexanderian's} work on the theory and practice such problems to be
suitable for me as well; should the department offer another inverse problems
course like they did in Spring 2019, I plan on taking it, if not, then
self-studying it. Finally, more towards the field of computational algebra, {\em
Professor Kaltofen's} work on sparse polynomial interpolation, and his act of
co-founding LinBox, which I am a fan of, interest me as well. For these reasons,
I would be very happy to join the mathematics department at North Carolina
State.

\end{document}
